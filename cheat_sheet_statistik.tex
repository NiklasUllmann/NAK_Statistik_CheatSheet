\documentclass[a4paper]{article}

\usepackage{pdflscape}
\usepackage{multicol}
\usepackage{blindtext}
\usepackage{color}
\usepackage{enumitem}

\usepackage[left=20mm, right=20mm, top=20mm, bottom=20mm]{geometry}



\setlength{\columnseprule}{0.5pt}
\def\columnseprulecolor{\color{black}}

\pagenumbering{gobble}

\title{Statistik Cheat Sheet}
\author{MADSo21}
\date{Fall 2021}


\begin{document}
\begin{landscape}
    \thispagestyle{empty}

    \begin{multicols}{4}
    
    \section{Beschreibende Statistkik}
        \begin{itemize}[noitemsep,nolistsep,leftmargin=*]
            \item Qualitative Merkmale: 
                \begin{itemize}[noitemsep,nolistsep,leftmargin=*]
                    \item Variieren nach Beschaffenheit
                    \item Bspw. Geschlecht
                \end{itemize}
            \item Quantitative Merkmale:
                \begin{itemize}[noitemsep,nolistsep]
                    \item Variieren nach Wert/Zahlen
                    \item Bspw. Alter, Einkommen
                \end{itemize}
        \end{itemize}
        \begin{itemize}[noitemsep,nolistsep,leftmargin=*]
            \item Diskrete Merkmale: 
                \begin{itemize}[noitemsep,nolistsep,leftmargin=*]
                    \item abgestufte Werte
                    \item Bspw. Einkommensklasse
                \end{itemize}
            \item Stetige Merkmale:
                \begin{itemize}[noitemsep,nolistsep,leftmargin=*]
                    \item können im Intervall jeden reellen Wert annehmen
                    \item Bspw. Körpergröße
                \end{itemize}
        \end{itemize}
    \subsubsection*{Skalenniveaus}
    \begin{itemize}[noitemsep,nolistsep,leftmargin=*]
        \item Nominal
        \begin{itemize}[noitemsep,nolistsep,leftmargin=*]
            \item nur Gleichheit oder Andersartigkeit feststellbar (keine Bewertung)
            \item stets qualitativ (Religion, Beruf etc.)
        \end{itemize}
        \item Ordinal
        \begin{itemize}[noitemsep,nolistsep,leftmargin=*]
            \item natürliche oder festzulegende Rangfolge
            \item IQ, Schulnoten
        \end{itemize}
        \item Kardinal
        \begin{itemize}[noitemsep,nolistsep,leftmargin=*]
            \item numerischer Art 
            \item Ausprägung und Unterschied sind messbar
            \item verhältnisskaliert (Absoluter Nullpunkt vorhanden; Gewicht, Preis (Doppelt so viel.))
            \item intervallskaliert (Kein Nullpunkt, nur Differenzen; Temperatur (10 Grad wärmer als gestern))
        \end{itemize}
    \end{itemize}
    \subsubsection*{Werte}
    \begin{itemize}[noitemsep,nolistsep,leftmargin=*]
        \item Arithmetisches Mittel $\overline{x}$
    \end{itemize}


    \section{Wahrscheinlich-keitsrechnung}
    Hallo
    \section{Schließende Statistik}
    Hallo
    \section{Taschenrechner}
    Hallo
    
    \end{multicols}
\end{landscape}
\end{document}