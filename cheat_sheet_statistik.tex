\documentclass[a4paper]{article}

\usepackage{pdflscape}
\usepackage{multicol}
\usepackage{blindtext}
\usepackage{color}
\usepackage{enumitem}

\usepackage[left=10mm, right=10mm, top=10mm, bottom=10mm]{geometry}



\setlength{\columnseprule}{0.5pt}
\def\columnseprulecolor{\color{black}}

\pagenumbering{gobble}

\title{Statistik Cheat Sheet}
\author{MADSo21}
\date{Fall 2021}


\begin{document}
\begin{landscape}
    \thispagestyle{empty}

    \begin{multicols}{4}
    \section{B. Statistkik}
        \begin{itemize}[noitemsep,nolistsep,leftmargin=*]
            \item Qualitative Merkmale: 
                \begin{itemize}[noitemsep,nolistsep,leftmargin=*]
                    \item Variieren nach Beschaffenheit
                    \item Bspw. Geschlecht
                \end{itemize}
            \item Quantitative Merkmale:
                \begin{itemize}[noitemsep,nolistsep]
                    \item Variieren nach Wert/Zahlen
                    \item Bspw. Alter, Einkommen
                \end{itemize}
        \end{itemize}
        \begin{itemize}[noitemsep,nolistsep,leftmargin=*]
            \item Diskrete Merkmale: 
                \begin{itemize}[noitemsep,nolistsep,leftmargin=*]
                    \item abgestufte Werte
                    \item Bspw. Einkommensklasse
                \end{itemize}
            \item Stetige Merkmale:
                \begin{itemize}[noitemsep,nolistsep,leftmargin=*]
                    \item können im Intervall jeden reellen Wert annehmen
                    \item Bspw. Körpergröße
                \end{itemize}
        \end{itemize}
    \subsubsection*{Skalenniveaus}
    \begin{itemize}[noitemsep,nolistsep,leftmargin=*]
        \item Nominal
        \begin{itemize}[noitemsep,nolistsep,leftmargin=*]
            \item nur Gleichheit oder Andersartigkeit feststellbar (keine Bewertung)
            \item stets qualitativ (Religion, Beruf etc.)
        \end{itemize}
        \item Ordinal
        \begin{itemize}[noitemsep,nolistsep,leftmargin=*]
            \item natürliche oder festzulegende Rangfolge
            \item IQ, Schulnoten
        \end{itemize}
        \item Kardinal
        \begin{itemize}[noitemsep,nolistsep,leftmargin=*]
            \item numerischer Art 
            \item Ausprägung und Unterschied sind messbar
            \item verhältnisskaliert (Absoluter Nullpunkt vorhanden; Gewicht, Preis (Doppelt so viel.))
            \item intervallskaliert (Kein Nullpunkt, nur Differenzen; Temperatur (10 Grad wärmer als gestern))
        \end{itemize}
    \end{itemize}
    \subsubsection*{Werte}
    \begin{itemize}[noitemsep,nolistsep,leftmargin=*]
        \item Arithmetisches Mittel $\overline{x}$
        \begin{itemize}[noitemsep,nolistsep,leftmargin=*]
            \item $\overline{x}=\frac{1}{n}\sum _{i=1}^{n}a_{i}={\frac {a_{1}+a_{2}+\cdots +a_{n}}{n}}$ 
            \item Summe aller Abweichungen vom Mittel = 0
            \item Verschiebung um kostanten Wert a $a + \overline{x}$
            \item Multiplikation mit konstantem Wert $a \cdot \overline{x}$
        \end{itemize}
        \item Median $\widetilde{x}$
        \begin{itemize}[noitemsep,nolistsep,leftmargin=*]
            \item Mittleres Element der geordneten Liste
            \item Bei gerader Anzahl, Durchschnitt der mittleren Elemente
        \end{itemize}
        \item Quartile (FEHLT)
        \begin{itemize}[noitemsep,nolistsep,leftmargin=*]
            \item Unteres Quartil $\widetilde{x}_{0,25}$
            \item Oberes Quartil $\widetilde{x}_{0,75}$
        \end{itemize}
        \item Varianz $\sigma^2$
        \begin{itemize}[noitemsep,nolistsep,leftmargin=*]
            \item Populations Varianz $\sigma^2 = \frac{\displaystyle\sum_{i=1}^{N}(x_i - \mu)^2} {N}$
            \item Sample Varianz $S^2_{n-1} = \frac{\displaystyle\sum_{i=1}^{n}(x_i - \overline{x})^2} {n-1}$
            \item Altn. Formel $\sigma^2 = \overline{x^2} - \overline{x}^2$
            \item Eigenschaften:
                \begin{itemize}[noitemsep,nolistsep,leftmargin=*]
                    \item Immer $>= 0$
                    \item Addition mit a, Varianz unverändert
                    \item Multiplikation mit b, $Varianz * b^2$
                \end{itemize}
        \end{itemize}
        \item Standardabweichung $\sigma$
        \begin{itemize}[noitemsep,nolistsep,leftmargin=*]
            \item $\sigma = \sqrt{\sigma^2}$
            \item StichprobenSTD $S = \sqrt{S^2_{n-1}}$
        \end{itemize}
        \item Quartilsabstand (FEHLT)
    \end{itemize}

    \subsubsection*{Zweidimensionale Häuffigkeitstabellen}
    \begin{itemize}[noitemsep,nolistsep,leftmargin=*]
        \item Statistische Variablen X und Y mit versch.Auspräungen
        \item Spaltensummen sowie Zeilensummen = n
        \item Relative Häufigkeit $h_{ij} = \frac{n_{ij}}{n}$
        \item Randverteilung = Betrachtung einer einzigen Variable
        \item $ Z = X +Y$; $\overline{z} = \overline{x} + \overline{y}$; 
    \end{itemize}

    \subsubsection*{Kovarianz}
    \begin{itemize}[noitemsep,nolistsep,leftmargin=*]
        \item Arithmetisches Mittel des Produkts der Abweichung der einzelnen Beobachtungen von ihrem Mittel
        \item $C_{XY} := \frac{1}{n}\sum_{j = 1}^{n}{(x_j - \overline{x})(y_j - \overline{y})}$ 
        \item $C_{XY} = \overline{xy} - \overline{x}*\overline{y}$
        \item  $C_{XY} > 0$ "große X-Werte zu großen Y-Werten"
        \item $C_{XY} < 0$ "große Werte zu kleine Werten"
        \item Sind zwei Variablen statistisch unabhängig ist die Kovarianz = 0
    \end{itemize}

    \subsubsection*{Korrelation}
    \begin{itemize}[noitemsep,nolistsep,leftmargin=*]
        \item Normal (Pearson) $r_{XY} = \frac{C_{XY}}{\sigma_x * \sigma_y}$
        \begin{itemize}[noitemsep,nolistsep,leftmargin=*]
            \item normiertes Maß für Strenge des linearen statistischen Zusammenhangs
            \item $r_{XY}$ hat das gleiche Vorzeichen wie $C_{XY}$
            \item Bleibt unverändert bei linearer Transformation
            \item $r_{XY} = r_{YX}$
        \end{itemize}
        \item Rangkorrelation (Spearman) $r_{XY}^{Sp} = r_{rg(X), rg(Y)}$
        \begin{itemize}[noitemsep,nolistsep,leftmargin=*]
            \item für ordinale Variablen
            \item misst monotonen Anteil des stat. Zusammenhangs
            \item Ränge müssen vorher berechnet werden
        \end{itemize}
        \item Kovarianz und Korrelation bedeuten nicht zwangsweise eine kausale Beziehung!
    \end{itemize}

    \subsubsection*{Kontingenzkoeffizient}
    \begin{itemize}[noitemsep,nolistsep,leftmargin=*]
        \item beschreibt die Stärke des Zusammenhangs zweier Merkmale, nicht deren Richtung
        \item Chi-Quadrat $QK = \sum_{i=1}^{k}\sum_{j=1}^{l}\frac{(n_{ji}-E_{ij})^2}{E_{ij}}$
        \begin{itemize}[noitemsep,nolistsep,leftmargin=*]
            \item $E_{ij} = \frac{1}{n}*n_i*n_j = \frac{1}{n}n(x_i)*n(y_j)$
            \item Siehe Erweiterte Kontingeztabelle
            \item X und Y unabhängig: $QK = 0$
            \item Sonst $QK > 0$
            \item Für 2x2 Matrix: $QK = \frac{n(ad-bc)^2}{(a+b)(a+c)(b+d)(c+d)}$
            \item a bis d sind Inhalte der Tabelle, Summen sind Randhäufigkeiten
            
        \end{itemize}
        \item Kontingenzkoeffizient $K := \sqrt{\frac{QK}{QK+n}}$
        \begin{itemize}[noitemsep,nolistsep,leftmargin=*]
            \item normiertes Maß
            \item X und Y unabhängig: $K = 0$
            \item $0 <= K <= K_{max} = \sqrt{\frac{m-1}{m}} < 1$
            \item m = Minimum von Zeilenzahl und Spaltenzahl
        \end{itemize}
        \item Korrigierter K.-koeffizient $K^* := \frac{K}{K_{max}} = \sqrt{\frac{QK*m}{(QK+n)(m-1)}}$
        \item \begin{itemize}[noitemsep,nolistsep,leftmargin=*]
            \item $ 0 <= K^* <= 1$
            \item Vergleichbar mit anderen K-Tabellen
        \end{itemize}
    \end{itemize}


  




    \section{Regression}

    \begin{itemize}[noitemsep,nolistsep,leftmargin=*]
        \item Lineare Regression $y(x) = a +bx$
        \item $b = \frac{c_{XY}}{s_X^2}$ und $a = \overline{y} - b\overline{x}$
    \end{itemize}


    \section{S. Statistik}
    Hallo
    \section{Taschenrechner}
    Hallo
    
    \end{multicols}

    \begin{table}[]
        \begin{tabular}{l|l}
        $n_{ij}$ & $(n_{ij}-E_{ij})^2$           \\ \hline
        $n_{ij}-E_{ij}$           & $E_{ij}$
        \end{tabular}
        \end{table}
\end{landscape}
\end{document}