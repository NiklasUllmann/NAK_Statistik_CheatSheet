\documentclass[a4paper]{article}

\usepackage{pdflscape}
\usepackage{multicol}
\usepackage{blindtext}
\usepackage{color}
\usepackage{enumitem}

\usepackage[left=10mm, right=10mm, top=10mm, bottom=10mm]{geometry}

\usepackage{titlesec}

\usepackage[utf8]{inputenc}
\usepackage{fourier} 
\usepackage{array}
\usepackage{makecell}

\renewcommand\theadalign{bc}
\renewcommand\theadfont{\bfseries}
\renewcommand\theadgape{\Gape[4pt]}
\renewcommand\cellgape{\Gape[4pt]}

\titlespacing\section{0pt}{10pt plus 4pt minus 2pt}{0pt plus 2pt minus 2pt}
\titlespacing\subsection{0pt}{10pt plus 4pt minus 2pt}{0pt plus 2pt minus 2pt}
\titlespacing\subsubsection{0pt}{10pt plus 4pt minus 2pt}{0pt plus 2pt minus 2pt}



\setlength{\columnseprule}{0.5pt}
\def\columnseprulecolor{\color{black}}

\pagenumbering{gobble}

\title{Statistik Cheat Sheet}
\author{MADSo21}
\date{Fall 2021}


\begin{document}
\begin{landscape}
    \thispagestyle{empty}

    \begin{multicols}{4}
        \begin{center}
            \begin{tabular}{ll|ll}
                Arith. Mittel & $\overline{x}$  & Kovarianz &  $C_{XY}$\\
                Median      & $\widetilde{x}$ &  Korrelation     & $r_{XY}$  \\
                Varianz      & $\sigma^2$ &  Chi Quadrat     &  $\chi^2$\\
                SDA      &  $\sigma$&  KontingenzK    & $K$ \\
                Sample Var.& $S^2$& Sample SDA& $S$ \\
                Bestimmtheit& $R^2$&Korrig. K& $K^*$\\
                Adj. Best. & $R^2_a$& Erwartungswert& $E(X)$
                \end{tabular}
        \end{center}
    \section{B. Statistik}
        \begin{itemize}[noitemsep,nolistsep,leftmargin=*]
            \item Qualitative Merkmale: 
                \begin{itemize}[noitemsep,nolistsep,leftmargin=*]
                    \item Variieren nach Beschaffenheit
                    \item Bspw. Geschlecht
                \end{itemize}
            \item Quantitative Merkmale:
                \begin{itemize}[noitemsep,nolistsep]
                    \item Variieren nach Wert/Zahlen
                    \item Bspw. Alter, Einkommen
                \end{itemize}
        \end{itemize}
        \begin{itemize}[noitemsep,nolistsep,leftmargin=*]
            \item Diskrete Merkmale: 
                \begin{itemize}[noitemsep,nolistsep,leftmargin=*]
                    \item abgestufte Werte
                    \item Bspw. Einkommensklasse
                \end{itemize}
            \item Stetige Merkmale:
                \begin{itemize}[noitemsep,nolistsep,leftmargin=*]
                    \item können im Intervall jeden reellen Wert annehmen
                    \item Bspw. Körpergröße
                \end{itemize}
        \end{itemize}
    \subsubsection*{Skalenniveaus}
    \begin{itemize}[noitemsep,nolistsep,leftmargin=*]
        \item Nominal
        \begin{itemize}[noitemsep,nolistsep,leftmargin=*]
            \item nur Gleichheit oder Andersartigkeit feststellbar (keine Bewertung)
            \item stets qualitativ (Religion, Beruf etc.)
        \end{itemize}
        \item Ordinal
        \begin{itemize}[noitemsep,nolistsep,leftmargin=*]
            \item natürliche oder festzulegende Rangfolge
            \item IQ, Schulnoten
        \end{itemize}
        \item Kardinal
        \begin{itemize}[noitemsep,nolistsep,leftmargin=*]
            \item numerischer Art 
            \item Ausprägung und Unterschied sind messbar
            \item verhältnisskaliert (Absoluter Nullpunkt vorhanden; Gewicht, Preis (Doppelt so viel.))
            \item intervallskaliert (Kein Nullpunkt, nur Differenzen; Temperatur (10 Grad wärmer als gestern))
        \end{itemize}
    \end{itemize}
    \subsubsection*{Werte}
    \begin{itemize}[noitemsep,nolistsep,leftmargin=*]
        \item Arithmetisches Mittel $\overline{x}$
        \begin{itemize}[noitemsep,nolistsep,leftmargin=*]
            \item $\overline{x}=\frac{1}{n}\sum _{i=1}^{n}a_{i}={\frac {a_{1}+a_{2}+\cdots +a_{n}}{n}}$ 
            \item Summe aller Abweichungen vom Mittel = 0
            \item Verschiebung um kostanten Wert a $a + \overline{x}$
            \item Multiplikation mit konstantem Wert $a \cdot \overline{x}$
        \end{itemize}
        \item Median $\widetilde{x}$
        \begin{itemize}[noitemsep,nolistsep,leftmargin=*]
            \item Mittleres Element der geordneten Liste
            \item Bei gerader Anzahl, Durchschnitt der mittleren Elemente
        \end{itemize}
        \item Quartile
        \begin{itemize}[noitemsep,nolistsep,leftmargin=*]
            \item Unteres Quartil $\widetilde{x}_{0,25}$ (sortieren \& ablesen)
            \item Oberes Quartil $\widetilde{x}_{0,75}$
        \end{itemize}
        \item Varianz $\sigma^2$
        \begin{itemize}[noitemsep,nolistsep,leftmargin=*]
            \item Populations Varianz $\sigma^2 = \frac{\displaystyle\sum_{i=1}^{N}(x_i - \mu)^2} {N}$
            \item Sample Varianz $S^2_{n-1} = \frac{\displaystyle\sum_{i=1}^{n}(x_i - \overline{x})^2} {n-1}$
            \item Altn. Formel $\sigma^2 = \overline{x^2} - \overline{x}^2$
            \item Eigenschaften:
                \begin{itemize}[noitemsep,nolistsep,leftmargin=*]
                    \item Immer $ \geq 0$
                    \item Addition mit a, Varianz unverändert
                    \item Multiplikation mit b, $Varianz * b^2$
                \end{itemize}
        \end{itemize}
        \item Standardabweichung $\sigma$
        \begin{itemize}[noitemsep,nolistsep,leftmargin=*]
            \item $\sigma = \sqrt{\sigma^2}$
            \item StichprobenSDA $S = \sqrt{S^2_{n-1}}$
        \end{itemize}
        \item Quartilsabstand $\widetilde{x}_{0,75} - \widetilde{x}_{0,25}$
    \end{itemize}

    \subsubsection*{Zweidimensionale Häuffigkeitstabellen}
    \begin{itemize}[noitemsep,nolistsep,leftmargin=*]
        \item Statistische Variablen X und Y mit versch.Auspräungen
        \item Spaltensummen sowie Zeilensummen = n
        \item Relative Häufigkeit $h_{ij} = \frac{n_{ij}}{n}$
        \item Randverteilung = Betrachtung einer einzigen Variable
        \item $ Z = X +Y$; $\overline{z} = \overline{x} + \overline{y}$; 
    \end{itemize}

    \subsubsection*{Kovarianz}
    \begin{itemize}[noitemsep,nolistsep,leftmargin=*]
        \item Arithmetisches Mittel des Produkts der Abweichung der einzelnen Beobachtungen von ihrem Mittel
        \item $C_{XY} := \frac{1}{n}\sum_{j = 1}^{n}{(x_j - \overline{x})(y_j - \overline{y})} = \overline{xy} - \overline{x}*\overline{y}$ 
        \item  $C_{XY} > 0$ "große X-Werte zu großen Y-Werten"
        \item $C_{XY} < 0$ "große Werte zu kleine Werten"
        \item Sind zwei Variablen statistisch unabhängig ist die Kovarianz = 0
    \end{itemize}

    \subsubsection*{Korrelation}
    \begin{itemize}[noitemsep,nolistsep,leftmargin=*]
        \item Normal (Pearson) $r_{XY} = \frac{C_{XY}}{\sigma_x * \sigma_y}$
        \begin{itemize}[noitemsep,nolistsep,leftmargin=*]
            \item normiertes Maß für Strenge des linearen statistischen Zusammenhangs
            \item $r_{XY}$ hat das gleiche Vorzeichen wie $C_{XY}$
            \item Bleibt unverändert bei linearer Transformation
            \item $r_{XY} = r_{YX}$
            \item $-1 \leq r_{XY} \leq +1$
        \end{itemize}
        \item Rangkorrelation (Spearman) $r_{XY}^{Sp} = r_{rg(X), rg(Y)}$
        \begin{itemize}[noitemsep,nolistsep,leftmargin=*]
            \item für ordinale Variablen
            \item misst monotonen Anteil des stat. Zusammenhangs
            \item Ränge müssen vorher berechnet werden
            \item $-1 \leq r_{XY}^{Sp} \leq +1$

        \end{itemize}
        \item Kovarianz und Korrelation bedeuten nicht zwangsweise eine kausale Beziehung!
    \end{itemize}

    \subsubsection*{Kontingenzkoeffizient}
    \begin{itemize}[noitemsep,nolistsep,leftmargin=*]
        \item beschreibt die Stärke des Zusammenhangs zweier Merkmale, nicht deren Richtung
        \item Chi-Quadrat $QK = \sum_{i=1}^{k}\sum_{j=1}^{l}\frac{(n_{ji}-E_{ij})^2}{E_{ij}}$
        \begin{itemize}[noitemsep,nolistsep,leftmargin=*]
            \item $E_{ij} = \frac{1}{n}*n_i*n_j = \frac{1}{n}n(x_i)*n(y_j)$
            \item Siehe Erweiterte Kontingeztabelle
            \item X und Y unabhängig: $QK = 0$
            \item Sonst $QK > 0$
            \item Für 2x2 Matrix: $QK = \frac{n(ad-bc)^2}{(a+b)(a+c)(b+d)(c+d)}$
            \item a bis d sind Inhalte der Tabelle, Summen sind Randhäufigkeiten
            
        \end{itemize}
        \item Kontingenzkoeffizient $K := \sqrt{\frac{QK}{QK+n}}$
        \begin{itemize}[noitemsep,nolistsep,leftmargin=*]
            \item normiertes Maß
            \item X und Y unabhängig: $K = 0$
            \item $0  \leq K  \leq K_{max} = \sqrt{\frac{m-1}{m}} < 1$
            \item m = Minimum von Zeilenzahl und Spaltenzahl
        \end{itemize}
        \item Korrigierter K.-koeffizient $K^* := \frac{K}{K_{max}} = \sqrt{\frac{QK*m}{(QK+n)(m-1)}}$
        \item \begin{itemize}[noitemsep,nolistsep,leftmargin=*]
            \item $ 0  \leq K^*  \leq 1$
            \item Vergleichbar mit anderen K-Tabellen
        \end{itemize}
    \end{itemize}


  




    \section{Regression}

    \begin{itemize}[noitemsep,nolistsep,leftmargin=*]
        \item Lineare Regression $y(x) = a +bx$
        \begin{itemize}[noitemsep,nolistsep,leftmargin=*]
            \item $b = \frac{c_{XY}}{s_X^2}$ und $a = \overline{y} - b\overline{x}$
            \item Interpret: b*x erhöht pro Einheit und a: Achsenabschnitt
            \item Extrapolation (Punkte außerhalb der orig. Daten) nicht aussagekräftig
            \item Regressionswerte = $\widehat{y}_i = y(x_i)$
            \item Residuen (Fehler) $e_i = y_i - \widehat{y}_i$
        \end{itemize}
        \item Andere Regressionen:
        \begin{itemize}[noitemsep,nolistsep,leftmargin=*]
            \item $\hat{y} =  a+bx+cx^2$ Quadr. Regr.
            \item $\hat{y} =  a+x^b$ Potenzfunkt.
            \item $\hat{y} =  ab^x$ Expo-funkt.
        \end{itemize}
        \item Meth. kleinste Quadrate
        \item Varianzzerlegung $SSQ_{Total} = SSQ_{Reg} + SSQ_{Resi}$
        \begin{itemize}[noitemsep,nolistsep,leftmargin=*]
            \item $SSQ_{Reg} = \sum^n_{i=1} (\hat{y}_i-\overline{y})^2$ (Abweichung von Vorhersage und Mittelwert)
            \item $SSQ_{Total} = \sum^n_{i=1} (y_i-\overline{y})^2$ (Gesamtabweichung)
            \item $SSQ_{Resi} = \sum^n_{i=1} (y_i-\hat{y}_i)^2$ (Abweichung von Vorhersage und y)
        \end{itemize}
        \item Bestimmtheitsmaß $R^2 = \frac{SSQ_{Reg}}{SSQ_{Total}} = \frac{S^2_{\hat{Y}}}{S^2_Y} = r^2$
        \begin{itemize}[noitemsep,nolistsep,leftmargin=*]
            \item $r^2$ gilt nicht für Quadr. Reg. !!!
            \item Schlecht $0  \leq R^2  \leq 1$ Gut 
            \item $R^2  \geq 0.8$ akzeptabel
        \end{itemize}
        \item Multiple Regr. 
        \begin{itemize}[noitemsep,nolistsep,leftmargin=*]
            \item Y wird durch mehrere Variablen erklärt
            \item $\hat{y} = a + b_1x_3 + b_2x_3 + b_3x_3$
        \end{itemize}
        \item Adjustiertes Bestimmtheitsmaß $R^2_{a} = R^2 - \frac{k}{n-k-1}*(1-R^2)$
        \begin{itemize}[noitemsep,nolistsep,leftmargin=*]
            \item Hinzunahme von Params, erhöht den $R^2$ automatisch, auch wenn es nicht besser wird
            \item $n = Anzahl der Messwerte$
            \item $k = Anzahl der Reg.Params$
            \item $R^2_{a}$ kann auch kleiner/negativ werden $->$ Variable nicht aufnehmen
        \end{itemize}
        \item Anmerkungen:
        \begin{itemize}[noitemsep,nolistsep,leftmargin=*]
            \item Residualplot: Gutes Modell, wenn kein Muster erkennbar!
            \item Optimum finden: 1.Ableitung = 0 setzen
            \item "Faktor Größe" hat nichts mit Einfluss zutun, nur bei standardisierten Daten
        \end{itemize}
    \end{itemize}

    \section{Wahrsch. Rech.}
    \begin{itemize}[noitemsep,nolistsep,leftmargin=*]
        \item Zufallsvariable $X : \Omega -> R\: mit\: X(\omega) = x$
        \begin{itemize}[noitemsep,nolistsep,leftmargin=*]
            \item Funktion, die jedem Möglichen Ergenis eine reelle Zahl zuordnet
            \item Wahrscheinlichkeits-/ Dichtefunktion $f: P(X =x)$
            \item Verteiteilungsfunktion $F: P(X  \leq t)$
            \item $F$ ist Stammfunktion für $f$ aber muss mit $+C$ angepasst werden
        \end{itemize}
        \item Diskrete
        \begin{itemize}[noitemsep,nolistsep,leftmargin=*]
            \item $f: \:R->[0,1] mit f(x)=P(X=x)$
            \item $P(X=X)$ Wahrscheinlichkeit mit der X die Realisation x annimmt
            \item $F(t) = P(X  \leq t) = \sum_{x_i  \leq t} P(X = x_i)$
        \end{itemize}
        \item Stetige
        \begin{itemize}[noitemsep,nolistsep,leftmargin=*]
            \item Zufallsvariable ist stetig, wenn Wahrscheinlichkeit durch Dichtefunktion abbilden lässt
            \item Dichtefunktion, wenn $\int^{+\infty}_{-\infty} f(x) dx = 1$ und $f(x)  \geq 0$
            \item $F(t) = P(X  \leq t) = \int^{t}_{-\infty} f(x)dx$
        \end{itemize}
        \item Erwartungswert
        \begin{itemize}[noitemsep,nolistsep,leftmargin=*]
            \item Diskret: $E(X) = \sum^n_{i=1} x_i * f(x_i)$
            \item Stetig: $E(X) = \int^{x_{max}}_{x_{min}} x * f(x) dx$
        \end{itemize}
        \item Varianz ($Var(X) = \sigma^2$) \& SDA ($\sigma = \sqrt{\sigma^2}$)
        \begin{itemize}[noitemsep,nolistsep,leftmargin=*]
            \item Es gilt: $\sigma^2 = E((X-E(X))^2) = E(X^2)- (E(X))^2$
            \item Diskret: $Var(X) = \sum^n_{i=1} (x_i-E(X))^2 * f(x_i)$
            \item Stetig: $Var(X) = \int^{x_{max}}_{x_{min}} (x-E(X))^2 * f(x) dx$
        \end{itemize}
        \item Rechenregeln
        \begin{itemize}[noitemsep,nolistsep,leftmargin=*]
            \item $E(a+b*X) = a + b* E(X)$
            \item $Var(a+b*X) = b^2 * Var(X)$
            \item $E(X+Y) = E(X) + E(Y)$
        \end{itemize}
        \item Stichprobe:
        \begin{itemize}[noitemsep,nolistsep,leftmargin=*]
            \item Stichprobenmittel von unabhängigen Variablen $\overline{X} := \frac{1}{n} (X_1 + \dots + X_n)$
            \item $E(\overline{X} = \mu)$ und $\sigma_{\overline{X}} = \frac{\sigma}{\sqrt{n}}$
        \end{itemize}
        \item Normalverteilung
        \begin{itemize}[noitemsep,nolistsep,leftmargin=*]
            \item SD-normalverteilung mit $\mu = 0$ und $\sigma = 1$
            \item z-Transformation $z = \frac{x-\mu}{\sigma}$
        \end{itemize}
        \item Zentr.Grenz.Satz: Für hinreichend großes n jeder Vertilung gilt $\overline{X}_n \tilde N(\mu, \frac{\sigma^2}{n}) $ "normalverteilt"
    \end{itemize}


    \section{Schl. Statistik}
        \subsubsection*{Anmerkungen}
        \begin{itemize}[noitemsep,nolistsep,leftmargin=*]
            \item $\alpha$ meist 5\% oder 1\%
        \end{itemize}
        \subsubsection*{Mittelwerttest}
        \begin{itemize}[noitemsep,nolistsep,leftmargin=*]
            \item GG ist norm. verteilt oder $n > 30$
            \item Stichprobenmittel $\overline{x}$ und ggf. Stichprobenvarianz $s^2$ bekannt
            \item $\sigma$ der GG bekannt
            \begin{itemize}[noitemsep,nolistsep,leftmargin=*]
                \item $z = \sqrt{n}\frac{\overline{x}-\mu_0}{\sigma}$
                \item $->$ Tabelle Norm.Verteilung
            \end{itemize}
            \item $\sigma$ der GG unbekannt
            \begin{itemize}[noitemsep,nolistsep,leftmargin=*]
                \item $t = \sqrt{n-1}\frac{\overline{x}-\mu_0}{s_n}$
                \item $>$ t Tabelle!
            \end{itemize}
            \item Gleiches gilt für t-1
        \end{itemize}
        \begin{center}
            \begin{tabular}{|l|l|l|}
             Seite:& $H_0$ behalten & $H_0$ verwerfen  \\ \hline
             Beide & $|z|  \leq z[1-\alpha/2]$  & $|z| > z[1-\alpha/2]$  \\ \hline
             Rechts& $z  \leq z[1-\alpha]$ &  $z > z[1-\alpha]$ \\ \hline
             Links & $z  \geq z[1-\alpha]$  &  $z < z[1-\alpha]$ \\ \hline
            \end{tabular}
        \end{center}




        \subsubsection*{Varianztest}
        \begin{itemize}[noitemsep,nolistsep,leftmargin=*]
            \item GG ist normalverteilt, $\alpha$ und $\sigma_0$ bekannt
            \item $\mu $ von GG. bekannt
            \item \begin{itemize}[noitemsep,nolistsep,leftmargin=*]
            \item $t_n= \frac{1}{\sigma^2_0} \sum^n_{i=1} (x_i-\mu)^2$
            \item Siehe (II)
            \end{itemize}
            \item $\mu $ von GG. unbekannt
            \begin{itemize}[noitemsep,nolistsep,leftmargin=*]
            \item $t_n= n*\frac{s^2_n}{\sigma^2_0}$
            \item Siehe (III)
            \end{itemize}
        \end{itemize}
        II
        \begin{center}
            \begin{tabular}{|l|l|l|}
            \hline
            $H_0$ & $H_1$ & Krit. \\ \hline
            $ \sigma^2 = \sigma^2_0$      & $ \sigma^2 \neq \sigma^2_0$ &\makecell{$t_n < \chi^2_n[\alpha/2]$\\$t_n > \chi^2_n[1-\alpha/2]$}\\ \hline
             $ \sigma^2  \geq \sigma^2_0$    & $\sigma^2 < \sigma^2_0$  &  $t_n < \chi^2_n[\alpha]$    \\ \hline
             $\sigma^2  \leq \sigma^2_0$     &  $\sigma^2 > \sigma^2_0$&   $t_n > \chi^2_n[1-\alpha]$   \\ \hline
            \end{tabular}
    \end{center}
        III
        \begin{center}
            \begin{tabular}{|l|l|l|}
            \hline
            $H_0$ & $H_1$ & Krit. \\ \hline
            $ \sigma^2 = \sigma^2_0$      & $ \sigma^2 \neq \sigma^2_0$ &\makecell{$t_n < \chi^2_{n-1}[\alpha/2]$\\$t_n > \chi^2_{n-1}[1-\alpha/2]$}\\ \hline
             $ \sigma^2  \geq \sigma^2_0$    & $\sigma^2 < \sigma^2_0$  &  $t_n < \chi^2_{n-1}[\alpha]$     \\ \hline
             $\sigma^2  \leq \sigma^2_0$     &  $\sigma^2 > \sigma^2_0$&   $t_n > \chi^2_{n-1}[1-\alpha]$   \\ \hline
            \end{tabular}
    \end{center}

        \subsubsection*{Differenztest}
        \begin{itemize}[noitemsep,nolistsep,leftmargin=*]
            \item GG ist normalverteilt
            \item $\sigma^2_X$ und $\sigma^2_Y$ gleich aber unbekannt
            \item $\delta_0$ vorgegeben oder $\delta = \mu_X - \mu_Y$
        \end{itemize}
             $$t = \frac{\overline{x}-\overline{y}-\delta_0}{\sqrt{\frac{1}{n}+\frac{1}{m}}* \sqrt{\frac{n*s^2_n + m*s^2_m}{n+m-2}}}$$
        \begin{center}
                \begin{tabular}{|l|l|l|}
                \hline
                $H_0$ & $H_1$ & Krit. \\ \hline
                $\delta=\delta_0$      & $ \delta \neq \delta_0$ &   $|t_n| > t_{n+m-2}[1-\alpha/2]$    \\ \hline
                 $\delta \geq\delta_0$     & $\delta < \delta_0$  &  $t_n < t_{n+m-2}[\alpha]$     \\ \hline
                 $\delta \leq\delta_0$     &  $\delta > \delta_0$&   $t_n > t_{n+m-2}[1-\alpha]$    \\ \hline
                \end{tabular}
        \end{center}
        
        \subsubsection*{$\chi^2$ Test}
        \begin{itemize}[noitemsep,nolistsep,leftmargin=*]
            \item $E_{ij} immer  \geq 5$
            \item $H_0$ = X, Y sind unabhängig; $H_1$ = X, Y sind abhängig 
            \item Prüfgröße $\chi^2$ (wie oben, mit erw. Kont.-Tabelle)
            \item Krit.Wert: $c = \chi^2_{(k-1)(l-1)}[1-\alpha]$
            \item $\chi^2  \leq c$ H0 behalten
            \item $\chi^2 > c$ H0 verwerfen
        \end{itemize}
        \subsubsection*{P-Test}

    \section*{Excel Tests}
    \begin{itemize}[noitemsep,nolistsep,leftmargin=*]
        \item Koeffizienten für jede $X_i$ $->$ Formel lässt sich daraus ableiten
        \item Parameter wird nur im Modell behalten wenn $t > |\frac{\hat{\beta_j}}{\hat{\sigma}_j}| > 2$
        \item Signifikanzniveau von ca. 5\%
        \item Alternativ: p-Werte < $\alpha$ werden behalten, p-Werte > $\alpha$ werden verworfen
        \item F-Test des Bestimmtheitsmaßes:
        \begin{itemize}[noitemsep,nolistsep,leftmargin=*]
            \item Testet ob, nicht auch alle Parameter = 0 sein könnten (Sinnhaftigkeit der Regression)
            \item Prüfgröße F aus Excel
            \item FWert: aus F-Verteilung oder gegeben
            \item $F  \geq FWert$ $H_0$ verwerfen, Regressionsansatz sinnvoll
            \item $F < FWert$ $H_0$ behalten, Regressionsansatz schlecht
            \item Einfacher: Über  F.Krit
            \item $p Wert < F.Krit$ $H_0$ behalten, Regressionsansatz sinnvoll
            \item $p Wert > F.Krit$ $H_0$ verwerfen, Regressionsansatz schlecht
        \end{itemize}
    \end{itemize}

    \section*{Other}
    \begin{center}
        \begin{tabular}{l|l}
            $n_{ij}$ & $(n_{ij}-E_{ij})^2$           \\ \hline
            $n_{ij}-E_{ij}$           & $E_{ij}$
            \end{tabular}
    \end{center}
    \begin{center}
        \begin{tabular}{|l|l|l|}
            Test\textbackslash{}Realität & $H_0$ richtig                           & $H_1$ richtig       \\ \hline
            $H_0$ behalten                  & \multicolumn{1}{l|}{ok (Spezifität)} & $\beta$ Fehler (FP)     \\ \hline 
            $H_0$ verwerfen                 & \multicolumn{1}{l|}{$\alpha$ Fehler (FN)}    & ok (Sensitivität) 
            \end{tabular}
    \end{center}
    
    \end{multicols}
\end{landscape}
\end{document}